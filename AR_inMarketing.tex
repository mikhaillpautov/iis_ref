Дополненная реальность в маркетинге "--- это интерактивный инструмент, который позволяет потребителям взаимодействовать с продуктами и брендами в формате смешанной реальности. Например, покупатели могут виртуально примерить одежду, увидеть, как будет выглядеть мебель в их доме, или получить дополнительную информацию о товаре, просто наведя камеру смартфона на упаковку.

Использование AR-технологий в маркетинговых кампаниях открывает ряд существенных преимуществ. Во"=первых, это повышение вовлечённости аудитории "--- интерактивный опыт запоминается лучше, чем традиционная реклама. Во"=вторых, AR помогает сократить разрыв между онлайн"= и офлайн"=покупками, позволяя клиентам лучше понять продукт до совершения покупки. В"=третьих, такой подход значительно увеличивает время взаимодействия потребителя с брендом.

Важно понимать различия между AR и другими цифровыми технологиями. В отличие от виртуальной реальности (VR), которая полностью погружает пользователя в цифровой мир, дополненная реальность накладывает цифровые элементы на реальное окружение. Смешанная реальность (MR) занимает промежуточное положение, позволяя цифровым объектам взаимодействовать с реальным миром. Каждая из этих технологий имеет свои преимущества, но именно AR стала наиболее доступной и практичной для маркетинговых целей, поскольку не требует специального оборудования, кроме смартфона.

\subsection{Виртуальные примерочные и 3D"=товары}
Дополненная реальность (AR) революционизирует процесс выбора и покупки товаров, позволяя потребителям взаимодействовать с продуктами в цифровом формате. Виртуальные примерочные и 3D"=модели товаров особенно востребованы в fashion"=индустрии, мебельном ритейле, косметике и электронике.

Технология использует компьютерное зрение, 3D"=моделирование и AR для наложения цифровых объектов на реальное изображение пользователя.

Технологии, лежащие в основе:
\begin{enumerate}
    \item \textbf{Face/body tracking} "--- отслеживание черт лица или фигуры для точной примерки (например, макияж или очки).
    \item \textbf{3D"=сканирование тела} "--- создание цифрового аватара для подбора одежды по параметрам.
    \item \textbf{AR"=наложение} "--- интеграция 3D"=модели товара в окружение (например, мебель в интерьере).
\end{enumerate}

Примеры внедрения:
\begin{enumerate}
    \item VK "--- мини"=приложение с нейросетью для 2D"=примерки (20 брендов: Love Republic, YOU WANNA, Studio 29 и др.)
    \item Lamoda "--- проект AR"=примерки обуви
    \item Nike "--- сервис Nike Fit (AR"=сканирование стопы для подбора обуви). 
    \item Prada "--- примерка солнцезащитных очков и аксессуаров с точным позиционированием.
    \item IKEA "--- приложение IKEA Place для размещения мебели в интерьере.
\end{enumerate}

Преимущеста AR"=примерочных:
\begin{itemize}
    \item упрощение и ускорение процесса покупки
    \item конкурентное преимущество на рынке
    \item снижение процента возвратов
    \item формирование чувства уверенности в покупке 
\end{itemize}



\subsection{Интерактивная реклама и упаковка}
Дополненная реальность превращает традиционную рекламу и упаковку в интерактивные цифровые платформы, создавая новый уровень вовлеченности потребителей. AR превращает статичную рекламу в динамичный диалог с потребителем. Новые технологии позволяют создавать эффект присутствия: например, при сканировании упаковки кофе пользователь может увидеть процесс его выращивания и обжарки.

Производители внедряют AR-метки на упаковку, которые при сканировании:
\begin{enumerate}
    \item Показывают дополнительную информацию о продукте.
    \item Запускают игры или викторины для повышения вовлечённости.
    \item Предлагают скидки и бонусы за взаимодействие.
\end{enumerate}
 
Примеры использования:
\begin{enumerate}
    \item МТС "--- <<Ожившие памятники>>. При наведении камеры смартфона на исторические здания (например, Большой театр) через приложение МТС запускался AR"=ролик с анимированной историей места.
    \item Coca"=Cola "--- анимированные персонажи на банках при сканировании.
    \item Pepsi "--- <<Невероятный>> автобусный щит (Лондон, 2014). Обычная остановка с рекламой Pepsi Max <<оживала>> при взгляде через AR"=экран. Прохожие видели, как из щита выбегает тигр, приземляется НЛО или нападает зомби.
    \item Nestle (бренд детского питания) "--- интерактивные сказки на упаковке. Сканирование упаковки запускало анимированную историю с персонажами.
    \item Toyota "--- интерактивные каталоги, где можно <<разобрать>> двигатель машины в AR.
\end{enumerate}

Таким образом, AR"=реклама и упаковка становятся не просто инструментами маркетинга, а частью удобного и экологичного взаимодействия с потребителем.

\subsection{AR в социальных сетях}
Соцсети становятся главной площадкой для AR"=экспериментов. Дополненная реальность совершила настоящую революцию в сфере развлекательного контента, перевернув традиционные представления о взаимодействии пользователей с цифровой средой. Этот технологический прорыв не просто добавил новые форматы, но изменил саму природу потребления контента "--- от пассивного наблюдения к активному соучастию. Современные социальные сети благодаря AR"=технологиям превратились в интерактивные площадки, где каждый пользователь становится одновременно и зрителем, и создателем, и участником цифрового действия.

Эволюция AR"=фильтров представляет собой увлекательный пример технологического прогресса. Если первые фильтры 2015"=2017 годов ограничивались простыми накладными элементами, то современные системы способны в реальном времени трансформировать черты лица, подстраиваясь под мимику и движения пользователя. Этот качественный скачок стал возможен благодаря развитию нейросетевых алгоритмов и компьютерного зрения.
