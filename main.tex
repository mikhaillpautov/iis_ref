\documentclass[referat]{SCWorks}
% Тип обучения (одно из значений):
%    bachelor   - бакалавриат (по умолчанию)
%    spec       - специальность
%    master     - магистратура
% Форма обучения (одно из значений):
%    och        - очное (по умолчанию)
%    zaoch      - заочное
% Тип работы (одно из значений):
%    coursework - курсовая работа (по умолчанию)
%    referat    - реферат
%  * otchet     - универсальный отчет
%  * nirjournal - журнал НИР
%  * digital    - итоговая работа для цифровой кафедры
%    diploma    - дипломная работа
%    pract      - отчет о научно-исследовательской работе
%    autoref    - автореферат выпускной работы
%    assignment - задание на выпускную квалификационную работу
%    review     - отзыв руководителя
%    critique   - рецензия на выпускную работу

% * Добавлены вручную. За вопросами к @mchernigin
\usepackage{preamble}

\begin{document}

% Кафедра (в родительном падеже)
\chair{математической кибернетики и компьютерных наук}

% Тема работы
\title{Дополненная реальность как инструмент маркетинга}

% Курс
\course{1}

% Группа
\group{151}

% Факультет (в родительном падеже) (по умолчанию "факультета КНиИТ")
% \department{факультета КНиИТ}

% Специальность/направление код - наименование
% \napravlenie{02.03.02 "--- Фундаментальная информатика и информационные технологии}
% \napravlenie{02.03.01 "--- Математическое обеспечение и администрирование информационных систем}
% \napravlenie{09.03.01 "--- Информатика и вычислительная техника}
\napravlenie{09.03.04 "--- Программная инженерия}
% \napravlenie{10.05.01 "--- Компьютерная безопасность}

% Для студентки. Для работы студента следующая команда не нужна.
 \studenttitle{студентов}

% Фамилия, имя, отчество в родительном падеже
\author{Ефимова Артёма Александровича \\ Паутова Михаила Сергеевича \\ Рыбалова Андрея Александровича}


% Заведующий кафедрой 
\chtitle{доцент, к.\,ф.-м.\,н.}
\chname{С.\,В.\,Миронов}

% Руководитель ДПП ПП для цифровой кафедры (перекрывает заведующего кафедры)
% \chpretitle{
%     заведующий кафедрой математических основ информатики и олимпиадного\\
%     программирования на базе МАОУ <<Ф"=Т лицей №1>>
% }
% \chtitle{г. Саратов, к.\,ф.-м.\,н., доцент}
% \chname{Кондратова\, Ю.\,Н.}

% Научный руководитель (для реферата преподаватель проверяющий работу)
\satitle{доцент, к. п. н.} %должность, степень, звание
\saname{А.\,П.\,Грецова}

% Руководитель практики от организации (руководитель для цифровой кафедры)
\patitle{доцент, к.\,ф.-м.\,н.}
\paname{С.\,В.\,Миронов}

% Руководитель НИР
\nirtitle{доцент, к.\,п.\,н.} % степень, звание
\nirname{В.\,А.\,Векслер}

% Семестр (только для практики, для остальных типов работ не используется)
\term{2}

% Наименование практики (только для практики, для остальных типов работ не
% используется)
\practtype{учебная}

% Продолжительность практики (количество недель) (только для практики, для
% остальных типов работ не используется)
\duration{2}

% Даты начала и окончания практики (только для практики, для остальных типов
% работ не используется)
\practStart{01.07.2024}
\practFinish{13.01.2024}

% Год выполнения отчета
\date{2025}

\maketitle

% Включение нумерации рисунков, формул и таблиц по разделам (по умолчанию -
% нумерация сквозная) (допускается оба вида нумерации)
\secNumbering

\tableofcontents

% Раздел "Обозначения и сокращения". Может отсутствовать в работе
% \abbreviations
% \begin{description}
%     \item ... "--- ...
%     \item ... "--- ...
% \end{description}

% Раздел "Определения". Может отсутствовать в работе
% \definitions

% Раздел "Определения, обозначения и сокращения". Может отсутствовать в работе.
% Если присутствует, то заменяет собой разделы "Обозначения и сокращения" и
% "Определения"
% \defabbr

\intro


Дополненная реальность (Augmented Reality, AR) представляет собой одну из наиболее перспективных технологий современности, стремительно трансформирующую способы взаимодействия человека с цифровой информацией. В отличие от виртуальной реальности, полностью погружающей пользователя в искусственно созданную среду, AR гармонично объединяет реальный и цифровой миры, накладывая компьютерные объекты и данные на физическое окружение в режиме реального времени. Эта технология, зародившаяся ещё в 1960"=х годах, сегодня переживает настоящий расцвет благодаря развитию мобильных устройств, совершенствованию компьютерного зрения и появлению новых аппаратных решений.

Современные AR"=решения находят применение в самых различных сферах человеческой деятельности: от образования и медицины до промышленности и розничной торговли. Особенно значимым представляется использование AR в маркетинге, где технология позволяет создавать принципиально новые форматы взаимодействия с потребителями. Виртуальные примерочные, интерактивные рекламные кампании, 3D"=визуализация товаров "--- всё это не только повышает вовлечённость аудитории, но и существенно трансформирует сам процесс принятия покупательских решений.

Однако, несмотря на очевидные преимущества, массовое внедрение AR"=технологий сталкивается с рядом существенных ограничений. К ним относятся технические барьеры, связанные с производительностью мобильных устройств, высокая стоимость разработки качественного AR"=контента, а также вопросы защиты персональных данных пользователей. Кроме того, отсутствие единых отраслевых стандартов и недостаток квалифицированных специалистов замедляют процесс коммерциализации технологии.

Целью данного реферата является комплексный анализ AR как технологической платформы, включая рассмотрение её базовых принципов работы, существующих и потенциальных областей применения в маркетинге, перспектив дальнейшего развития, а также ключевых проблем, препятствующих её повсеместному распространению. Такой всесторонний подход позволит объективно оценить текущее состояние технологии и её потенциал для трансформации маркетинговых коммуникаций в ближайшем будущем.


\section{Технологические основы AR}
Прежде чем рассматривать применение дополненной реальности в маркетинге и других сферах, необходимо разобраться в её технической основе. Современные AR"=системы представляют собой сложные программно"=аппаратные комплексы, использующие достижения компьютерного зрения, машинного обучения и 3D"=моделирования. В данном разделе мы рассмотрим фундаментальные принципы работы AR, ключевые технологии, обеспечивающие её функционирование, а также существующие типы и классификации систем дополненной реальности. Этот анализ позволит лучше понять как текущие возможности технологии, так и её ограничения, что особенно важно для оценки перспектив её коммерческого использования.

\subsection{Что такое дополненная реальность?}
Дополненная реальность (AR) представляет собой технологию, которая в реальном времени объединяет цифровые данные с физическим окружением пользователя. В отличие от виртуальной реальности, создающей полностью искусственную среду, AR лишь дополняет реальный мир компьютерными элементами "--- 3D"=моделями, текстовой информацией, анимацией или визуальными эффектами.

Ключевыми особенностями дополненной реальности являются:
\begin{enumerate}
	\item Интерактивность "--- пользователи могут взаимодействовать с цифровыми элементами через сенсорные экраны, голосовые команды или жесты. Например, в приложениях виртуальной примерки пользователи могут "примерять" одежду или аксессуары, изменяя их размер и положение.;
	\item Пространственная регистрация "--- цифровые объекты точно привязываются к реальным поверхностям и объектам, сохраняя свое положение при перемещении пользователя. Это достигается благодаря сложным алгоритмам компьютерного зрения и пространственного трекинга.;
	\item Реальномасштабная интеграция "--- виртуальные элементы учитывают физические параметры окружающей среды (освещение, перспективу, масштаб), что создает эффект их реального присутствия в пространстве;
	\item Обратная связь в реальном времени "--- система мгновенно реагирует на изменения в окружающей среде и действия пользователя, обеспечивая плавное и естественное взаимодействие.
\end{enumerate}

Технология AR находит применение в различных сферах: от развлекательных приложений и маркетинга до промышленного дизайна и медицины. Ее развитие стало возможным благодаря прогрессу в области мобильных технологий, компьютерного зрения и машинного обучения, что позволило создавать все более реалистичные и функциональные AR-решения.

\subsection {Как работает дополненная реальность?}

Принцип работы дополненной реальности основан на сложном взаимодействии нескольких ключевых технологических компонентов:
\begin{enumerate}
	\item Компьютерное зрение "--- фундаментальная технология, позволяющая системе "видеть" и анализировать окружающее пространство. Она включает:
		\begin{itemize}
			\item Распознавание объектов и плоскостей (стены, пол, мебель);
			\item Обнаружение маркеров (специальных меток или QR-кодов);
			\item Идентификация особенностей окружения (углы, текстуры, контуры).
		\end{itemize}		
	\item Пространственный трекинг "--- технология, отвечающая за:
		\begin{itemize}
			\item Определение положения и ориентации устройства в пространстве;
			\item Отслеживание перемещений пользователя;
			\item Привязку виртуальных объектов к реальным координатам.
		\end{itemize}
	\item  3D-рендеринг "--- процесс визуализации цифровых объектов, который:
		\begin{itemize}
			\item Обеспечивает реалистичное отображение 3D-моделей;
			\item Корректирует перспективу и масштаб в реальном времени;
			\item Учитывает освещение и тени для естественной интеграции.
		\end{itemize}
\end{enumerate}

Современные технологии дополненной реальности можно разделить на несколько ключевых типов, различающихся по принципам интеграции виртуального и физического миров.
\begin{enumerate}
	\item Маркерные системы основаны на распознавании специальных графических меток, что обеспечивает высокую точность позиционирования. Они широко применяются в промышленности и образовательных проектах, где требуется точное совмещение реальных и цифровых объектов.
	\item Безмаркерные SLAM"=системы (Simultaneous Localization and Mapping) не требуют специальных меток, а создают цифровую карту пространства в реальном времени, анализируя данные с камеры и датчиков устройства. Этот подход стал основой для большинства современных мобильных AR"=приложений.
	\item Проекционные системы используют специальные проекторы для отображения цифрового контента непосредственно на физические объекты. Такие решения часто встречаются в музеях и на выставках, создавая эффектные интерактивные инсталляции.
	\item Наложение на реальность "--- наиболее распространенный тип, применяемый в социальных сетях и маркетинговых кампаниях. Он не требует сложного трекинга окружения, просто добавляя виртуальные элементы к изображению с камеры.
\end{enumerate}

Эволюция AR-технологий привела к появлению гибридных систем, сочетающих преимущества разных подходов для создания более совершенных и универсальных решений.

\subsection{Оборудование для дополненной реальности} 

Дополненная реальность реализуется через три типа устройств:
\begin{enumerate}
	\item Смартфоны/планшеты "--- массовые AR-устройства с камерами и датчиками движения. Их преимущество "--- доступность для обычных пользователей;
	\item Спецгарнитуры (HoloLens, Magic Leap) "--- профессиональные решения с расширенными возможностями трекинга и hands"=free управлением;
	\item Дополнительные сенсоры (LiDAR, проекторы) "--- повышают точность AR-систем и расширяют сферы применения.
\end{enumerate}

Развитие AR-оборудования направлено на улучшение точности, удобства и доступности технологии. Современные тенденции - миниатюризация устройств и расширение их функционала.


\section{Применение AR в маркетинге}
Дополненная реальность в маркетинге "--- это интерактивный инструмент, который позволяет потребителям взаимодействовать с продуктами и брендами в формате смешанной реальности. Например, покупатели могут виртуально примерить одежду, увидеть, как будет выглядеть мебель в их доме, или получить дополнительную информацию о товаре, просто наведя камеру смартфона на упаковку.

Использование AR-технологий в маркетинговых кампаниях открывает ряд существенных преимуществ. Во"=первых, это повышение вовлечённости аудитории "--- интерактивный опыт запоминается лучше, чем традиционная реклама. Во"=вторых, AR помогает сократить разрыв между онлайн"= и офлайн"=покупками, позволяя клиентам лучше понять продукт до совершения покупки. В"=третьих, такой подход значительно увеличивает время взаимодействия потребителя с брендом.

Важно понимать различия между AR и другими цифровыми технологиями. В отличие от виртуальной реальности (VR), которая полностью погружает пользователя в цифровой мир, дополненная реальность накладывает цифровые элементы на реальное окружение. Смешанная реальность (MR) занимает промежуточное положение, позволяя цифровым объектам взаимодействовать с реальным миром. Каждая из этих технологий имеет свои преимущества, но именно AR стала наиболее доступной и практичной для маркетинговых целей, поскольку не требует специального оборудования, кроме смартфона.

\subsection{Виртуальные примерочные и 3D"=товары}
Дополненная реальность (AR) революционизирует процесс выбора и покупки товаров, позволяя потребителям взаимодействовать с продуктами в цифровом формате. Виртуальные примерочные и 3D"=модели товаров особенно востребованы в fashion"=индустрии, мебельном ритейле, косметике и электронике.

Технология использует компьютерное зрение, 3D"=моделирование и AR для наложения цифровых объектов на реальное изображение пользователя.

Технологии, лежащие в основе:
\begin{enumerate}
    \item \textbf{Face/body tracking} "--- отслеживание черт лица или фигуры для точной примерки (например, макияж или очки).
    \item \textbf{3D"=сканирование тела} "--- создание цифрового аватара для подбора одежды по параметрам.
    \item \textbf{AR"=наложение} "--- интеграция 3D"=модели товара в окружение (например, мебель в интерьере).
\end{enumerate}

Примеры внедрения:
\begin{enumerate}
    \item VK "--- мини"=приложение с нейросетью для 2D"=примерки (20 брендов: Love Republic, YOU WANNA, Studio 29 и др.)
    \item Lamoda "--- проект AR"=примерки обуви
    \item Nike "--- сервис Nike Fit (AR"=сканирование стопы для подбора обуви). 
    \item Prada "--- примерка солнцезащитных очков и аксессуаров с точным позиционированием.
    \item IKEA "--- приложение IKEA Place для размещения мебели в интерьере.
\end{enumerate}

Преимущеста AR"=примерочных:
\begin{itemize}
    \item упрощение и ускорение процесса покупки
    \item конкурентное преимущество на рынке
    \item снижение процента возвратов
    \item формирование чувства уверенности в покупке 
\end{itemize}



\subsection{Интерактивная реклама и упаковка}
Дополненная реальность превращает традиционную рекламу и упаковку в интерактивные цифровые платформы, создавая новый уровень вовлеченности потребителей. AR превращает статичную рекламу в динамичный диалог с потребителем. Новые технологии позволяют создавать эффект присутствия: например, при сканировании упаковки кофе пользователь может увидеть процесс его выращивания и обжарки.

Производители внедряют AR-метки на упаковку, которые при сканировании:
\begin{enumerate}
    \item Показывают дополнительную информацию о продукте.
    \item Запускают игры или викторины для повышения вовлечённости.
    \item Предлагают скидки и бонусы за взаимодействие.
\end{enumerate}
 
Примеры использования:
\begin{enumerate}
    \item МТС "--- <<Ожившие памятники>>. При наведении камеры смартфона на исторические здания (например, Большой театр) через приложение МТС запускался AR"=ролик с анимированной историей места.
    \item Coca"=Cola "--- анимированные персонажи на банках при сканировании.
    \item Pepsi "--- <<Невероятный>> автобусный щит (Лондон, 2014). Обычная остановка с рекламой Pepsi Max <<оживала>> при взгляде через AR"=экран. Прохожие видели, как из щита выбегает тигр, приземляется НЛО или нападает зомби.
    \item Nestle (бренд детского питания) "--- интерактивные сказки на упаковке. Сканирование упаковки запускало анимированную историю с персонажами.
    \item Toyota "--- интерактивные каталоги, где можно <<разобрать>> двигатель машины в AR.
\end{enumerate}

Таким образом, AR"=реклама и упаковка становятся не просто инструментами маркетинга, а частью удобного и экологичного взаимодействия с потребителем.

\subsection{AR в социальных сетях}
Соцсети становятся главной площадкой для AR"=экспериментов. Дополненная реальность совершила настоящую революцию в сфере развлекательного контента, перевернув традиционные представления о взаимодействии пользователей с цифровой средой. Этот технологический прорыв не просто добавил новые форматы, но изменил саму природу потребления контента "--- от пассивного наблюдения к активному соучастию. Современные социальные сети благодаря AR"=технологиям превратились в интерактивные площадки, где каждый пользователь становится одновременно и зрителем, и создателем, и участником цифрового действия.

Эволюция AR"=фильтров представляет собой увлекательный пример технологического прогресса. Если первые фильтры 2015"=2017 годов ограничивались простыми накладными элементами, то современные системы способны в реальном времени трансформировать черты лица, подстраиваясь под мимику и движения пользователя. Этот качественный скачок стал возможен благодаря развитию нейросетевых алгоритмов и компьютерного зрения.


\section{Перпективы развития AR}
Дополненная реальность (AR) продолжает активно развиваться, и среди наиболее перспективных направлений можно выделить WebAR, интеграцию с метавселенными и использование искусственного интеллекта (AI).

\subsection{WebAR}
WebAR позволяет интегрировать дополненную реальность в веб-сайты и приложения. 
Это достигается за счет использования современных API, таких как WebXR, и оптимизированных 3D"=движков. Благодаря WebAR, пользователи могут видеть виртуальные объекты, наложенные на реальное изображение через камеру своего смартфона или планшета. Это делает AR более доступной и удобной, чем раньше, поскольку нет необходимости скачивать и устанавливать специальные приложения. WebAR позволяет создавать интерактивные маркетинговые кампании, виртуальные экскурсии, интерактивные инструкции и многое другое.

Ключевые преимущества WebAR:
\begin{enumerate}
    \item Доступность "--- работает на большинстве смартфонов (iOS и Android) без дополнительных установок.
    \item Мгновенный доступ "--- пользователи могут взаимодействовать с AR-контентом через простую ссылку.
    \item Снижение барьеров для бизнеса "--- компании могут внедрять AR в маркетинг без разработки нативных приложений. 
\end{enumerate}

Однако, несмотря на все преимущества, WebAR еще находится на стадии развития. Не все веб-браузеры полностью поддерживают эти стандарты, а разработка качественного AR"=контента требует специальных навыков и знаний. Также существуют ограничения в производительности, которые могут приводить к проблемам с плавностью работы приложений на некоторых устройствах.

Тем не менее, WebAR представляет собой важный шаг в направлении более широкого распространения AR технологий. По мере совершенствования веб"=стандартов и улучшения производительности устройств, можно ожидать еще более широкого применения WebAR в различных сферах, включая веб"=разработку, маркетинг, образование и развлечения.

\subsection{Интеграция AR с метавселенными}
Метавселенные (Meta, Roblox, Decentraland) "--- это объединённые сетью глобально доступные виртуальные пространства, представляющие собой онлайн"=миры, созданные пользователями или компаниями. Эти пространства позволяют участникам общаться друг с другом, играть, учиться, торговать товарами и услугами, инвестировать и получать доход, используя собственные аватары и ресурсы.

Совместное внедрение AR и метавселенных позволит создать принципиально новую экосистему, обеспечивающую уникальный опыт взаимодействия между людьми и миром цифровой информации.

Важнейшей задачей при интеграции AR и метавселенных является разработка удобного интерфейса, минимизирующего когнитивную нагрузку на пользователя. Для этого необходимы продвинутые системы распознавания речи, жестов и взгляда, способные точно интерпретировать намерения пользователя и быстро реагировать на запросы.

Уже сейчас крупные технологические гиганты ведут разработки умных очков, поддерживающих AR"=функциональность и оснащённых датчиками глубины, камеры и голосового управления. Такие устройства будут способны интегрироваться с существующими системами метавселенных, предлагая удобные и интуитивные способы взаимодействия.

Среди известных примеров интеграции AR и метавселенных выделяются проекты таких компаний, как:
\begin{enumerate}
    \item Meta: Компания развивает платформу Horizon Worlds, предлагающую комплексный интерфейс, позволяющий создавать собственные VR/AR"=пространства внутри единой платформы.
    \item Microsoft: Проект Microsoft Mesh нацелен на создание гибридных виртуальных миров, куда участники смогут входить через гарнитуру HoloLens или любой другой AR"=гаджет.
    \item Google: Корпорация также участвует в разработке собственных решений для метавселенных, интегрируя их с картографическими сервисами и инструментами дополненной реальности.
\end{enumerate}

\subsection{Использование AI для улучшения AR}
Искусственный интеллект (AI) играет ключевую роль в развитии дополненной реальности, делая её более умной, адаптивной и персонализированной. С помощью AI AR"=системы могут анализировать окружающую среду в реальном времени, распознавать объекты, лица и жесты, а также предсказывать действия пользователя. Например, нейросети позволяют улучшить трекинг движений, что особенно важно для игр и интерактивных приложений. Кроме того, AI помогает создавать более реалистичные 3D"=модели и анимации, автоматически адаптируя контент под условия освещения и перспективу.

Одним из перспективных направлений является генеративный AI, который может создавать AR"=контент на лету, например, генерировать уникальные виртуальные объекты или даже целые сцены на основе текстовых запросов пользователя. Компании уже начинают внедрять такие решения: NVIDIA с проектом Omniverse использует AI для создания динамичных AR"=сред, а OpenAI исследует возможности интеграции GPT"=моделей в AR"=интерфейсы для более естественного взаимодействия.

Ещё одним важным аспектом является использование AI для анализа поведения пользователей. Это позволяет настраивать AR"=опыт индивидуально под каждого человека, предлагая релевантный контент и упрощая навигацию. Например, в розничной торговле AI может анализировать предпочтения покупателя и показывать ему персонализированные AR-рекомендации прямо в магазине.

Таким образом, сочетание AI и AR открывает новые горизонты для индустрии развлечений, образования, медицины и бизнеса, делая взаимодействие с цифровым миром более естественным и эффективным.


\conclusion
fdghjkl;'f w
wewerw
wrwr

ewrwer
werw
r

ewrwr
w
rwr
% Отобразить все источники. Даже те, на которые нет ссылок.
\nocite{*}

\bibliographystyle{ugost2003}
\bibliography{thesis}


% Окончание основного документа и начало приложений Каждая последующая секция
% документа будет являться приложением
\appendix

\end{document}
