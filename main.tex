\documentclass[referat]{SCWorks}
% Тип обучения (одно из значений):
%    bachelor   - бакалавриат (по умолчанию)
%    spec       - специальность
%    master     - магистратура
% Форма обучения (одно из значений):
%    och        - очное (по умолчанию)
%    zaoch      - заочное
% Тип работы (одно из значений):
%    coursework - курсовая работа (по умолчанию)
%    referat    - реферат
%  * otchet     - универсальный отчет
%  * nirjournal - журнал НИР
%  * digital    - итоговая работа для цифровой кафедры
%    diploma    - дипломная работа
%    pract      - отчет о научно-исследовательской работе
%    autoref    - автореферат выпускной работы
%    assignment - задание на выпускную квалификационную работу
%    review     - отзыв руководителя
%    critique   - рецензия на выпускную работу

% * Добавлены вручную. За вопросами к @mchernigin
\usepackage{preamble}

\begin{document}

% Кафедра (в родительном падеже)
\chair{математической кибернетики и компьютерных наук}

% Тема работы
\title{Дополненная реальность как инструмент маркетинга}

% Курс
\course{1}

% Группа
\group{151}

% Факультет (в родительном падеже) (по умолчанию "факультета КНиИТ")
% \department{факультета КНиИТ}

% Специальность/направление код - наименование
% \napravlenie{02.03.02 "--- Фундаментальная информатика и информационные технологии}
% \napravlenie{02.03.01 "--- Математическое обеспечение и администрирование информационных систем}
% \napravlenie{09.03.01 "--- Информатика и вычислительная техника}
\napravlenie{09.03.04 "--- Программная инженерия}
% \napravlenie{10.05.01 "--- Компьютерная безопасность}

% Для студентки. Для работы студента следующая команда не нужна.
% \studenttitle{студентки}

% Фамилия, имя, отчество в родительном падеже
\author{Абоба}


% Заведующий кафедрой 
\chtitle{доцент, к.\,ф.-м.\,н.}
\chname{С.\,В.\,Миронов}

% Руководитель ДПП ПП для цифровой кафедры (перекрывает заведующего кафедры)
% \chpretitle{
%     заведующий кафедрой математических основ информатики и олимпиадного\\
%     программирования на базе МАОУ <<Ф"=Т лицей №1>>
% }
% \chtitle{г. Саратов, к.\,ф.-м.\,н., доцент}
% \chname{Кондратова\, Ю.\,Н.}

% Научный руководитель (для реферата преподаватель проверяющий работу)
\satitle{доцент, } %должность, степень, звание
\saname{А.\,П.\,Грецова}

% Руководитель практики от организации (руководитель для цифровой кафедры)
\patitle{доцент, к.\,ф.-м.\,н.}
\paname{С.\,В.\,Миронов}

% Руководитель НИР
\nirtitle{доцент, к.\,п.\,н.} % степень, звание
\nirname{В.\,А.\,Векслер}

% Семестр (только для практики, для остальных типов работ не используется)
\term{2}

% Наименование практики (только для практики, для остальных типов работ не
% используется)
\practtype{учебная}

% Продолжительность практики (количество недель) (только для практики, для
% остальных типов работ не используется)
\duration{2}

% Даты начала и окончания практики (только для практики, для остальных типов
% работ не используется)
\practStart{01.07.2024}
\practFinish{13.01.2024}

% Год выполнения отчета
\date{2025}

\maketitle

% Включение нумерации рисунков, формул и таблиц по разделам (по умолчанию -
% нумерация сквозная) (допускается оба вида нумерации)
\secNumbering

\tableofcontents

% Раздел "Обозначения и сокращения". Может отсутствовать в работе
% \abbreviations
% \begin{description}
%     \item ... "--- ...
%     \item ... "--- ...
% \end{description}

% Раздел "Определения". Может отсутствовать в работе
% \definitions

% Раздел "Определения, обозначения и сокращения". Может отсутствовать в работе.
% Если присутствует, то заменяет собой разделы "Обозначения и сокращения" и
% "Определения"
% \defabbr

\intro


Дополненная реальность (Augmented Reality, AR) представляет собой одну из наиболее перспективных технологий современности, стремительно трансформирующую способы взаимодействия человека с цифровой информацией. В отличие от виртуальной реальности, полностью погружающей пользователя в искусственно созданную среду, AR гармонично объединяет реальный и цифровой миры, накладывая компьютерные объекты и данные на физическое окружение в режиме реального времени. Эта технология, зародившаяся ещё в 1960"=х годах, сегодня переживает настоящий расцвет благодаря развитию мобильных устройств, совершенствованию компьютерного зрения и появлению новых аппаратных решений.

Современные AR"=решения находят применение в самых различных сферах человеческой деятельности: от образования и медицины до промышленности и розничной торговли. Особенно значимым представляется использование AR в маркетинге, где технология позволяет создавать принципиально новые форматы взаимодействия с потребителями. Виртуальные примерочные, интерактивные рекламные кампании, 3D"=визуализация товаров "--- всё это не только повышает вовлечённость аудитории, но и существенно трансформирует сам процесс принятия покупательских решений.

Однако, несмотря на очевидные преимущества, массовое внедрение AR"=технологий сталкивается с рядом существенных ограничений. К ним относятся технические барьеры, связанные с производительностью мобильных устройств, высокая стоимость разработки качественного AR"=контента, а также вопросы защиты персональных данных пользователей. Кроме того, отсутствие единых отраслевых стандартов и недостаток квалифицированных специалистов замедляют процесс коммерциализации технологии.

Целью данного реферата является комплексный анализ AR как технологической платформы, включая рассмотрение её базовых принципов работы, существующих и потенциальных областей применения в маркетинге, перспектив дальнейшего развития, а также ключевых проблем, препятствующих её повсеместному распространению. Такой всесторонний подход позволит объективно оценить текущее состояние технологии и её потенциал для трансформации маркетинговых коммуникаций в ближайшем будущем.


\section{Технологические основы AR}
Прежде чем рассматривать применение дополненной реальности в маркетинге и других сферах, необходимо разобраться в её технической основе. Современные AR"=системы представляют собой сложные программно"=аппаратные комплексы, использующие достижения компьютерного зрения, машинного обучения и 3D"=моделирования. В данном разделе мы рассмотрим фундаментальные принципы работы AR, ключевые технологии, обеспечивающие её функционирование, а также существующие типы и классификации систем дополненной реальности. Этот анализ позволит лучше понять как текущие возможности технологии, так и её ограничения, что особенно важно для оценки перспектив её коммерческого использования.

\subsection{Что такое дополненная реальность?}
Дополненная реальность (AR) представляет собой технологию, которая в реальном времени объединяет цифровые данные с физическим окружением пользователя. В отличие от виртуальной реальности, создающей полностью искусственную среду, AR лишь дополняет реальный мир компьютерными элементами "--- 3D"=моделями, текстовой информацией, анимацией или визуальными эффектами.

Ключевыми особенностями дополненной реальности являются:
\begin{enumerate}
	\item Интерактивность "--- пользователи могут взаимодействовать с цифровыми элементами через сенсорные экраны, голосовые команды или жесты. Например, в приложениях виртуальной примерки пользователи могут "примерять" одежду или аксессуары, изменяя их размер и положение.;
	\item Пространственная регистрация "--- цифровые объекты точно привязываются к реальным поверхностям и объектам, сохраняя свое положение при перемещении пользователя. Это достигается благодаря сложным алгоритмам компьютерного зрения и пространственного трекинга.;
	\item Реальномасштабная интеграция "--- виртуальные элементы учитывают физические параметры окружающей среды (освещение, перспективу, масштаб), что создает эффект их реального присутствия в пространстве;
	\item Обратная связь в реальном времени "--- система мгновенно реагирует на изменения в окружающей среде и действия пользователя, обеспечивая плавное и естественное взаимодействие.
\end{enumerate}

Технология AR находит применение в различных сферах: от развлекательных приложений и маркетинга до промышленного дизайна и медицины. Ее развитие стало возможным благодаря прогрессу в области мобильных технологий, компьютерного зрения и машинного обучения, что позволило создавать все более реалистичные и функциональные AR-решения.

\subsection {Как работает дополненная реальность?}

Принцип работы дополненной реальности основан на сложном взаимодействии нескольких ключевых технологических компонентов:
\begin{enumerate}
	\item Компьютерное зрение "--- фундаментальная технология, позволяющая системе "видеть" и анализировать окружающее пространство. Она включает:
		\begin{itemize}
			\item Распознавание объектов и плоскостей (стены, пол, мебель);
			\item Обнаружение маркеров (специальных меток или QR-кодов);
			\item Идентификация особенностей окружения (углы, текстуры, контуры).
		\end{itemize}		
	\item Пространственный трекинг "--- технология, отвечающая за:
		\begin{itemize}
			\item Определение положения и ориентации устройства в пространстве;
			\item Отслеживание перемещений пользователя;
			\item Привязку виртуальных объектов к реальным координатам.
		\end{itemize}
	\item  3D-рендеринг "--- процесс визуализации цифровых объектов, который:
		\begin{itemize}
			\item Обеспечивает реалистичное отображение 3D-моделей;
			\item Корректирует перспективу и масштаб в реальном времени;
			\item Учитывает освещение и тени для естественной интеграции.
		\end{itemize}
\end{enumerate}

Современные технологии дополненной реальности можно разделить на несколько ключевых типов, различающихся по принципам интеграции виртуального и физического миров.
\begin{enumerate}
	\item Маркерные системы основаны на распознавании специальных графических меток, что обеспечивает высокую точность позиционирования. Они широко применяются в промышленности и образовательных проектах, где требуется точное совмещение реальных и цифровых объектов.
	\item Безмаркерные SLAM"=системы (Simultaneous Localization and Mapping) не требуют специальных меток, а создают цифровую карту пространства в реальном времени, анализируя данные с камеры и датчиков устройства. Этот подход стал основой для большинства современных мобильных AR"=приложений.
	\item Проекционные системы используют специальные проекторы для отображения цифрового контента непосредственно на физические объекты. Такие решения часто встречаются в музеях и на выставках, создавая эффектные интерактивные инсталляции.
	\item Наложение на реальность "--- наиболее распространенный тип, применяемый в социальных сетях и маркетинговых кампаниях. Он не требует сложного трекинга окружения, просто добавляя виртуальные элементы к изображению с камеры.
\end{enumerate}

Эволюция AR-технологий привела к появлению гибридных систем, сочетающих преимущества разных подходов для создания более совершенных и универсальных решений.

\subsection{Оборудование для дополненной реальности} 

Дополненная реальность реализуется через три типа устройств:
\begin{enumerate}
	\item Смартфоны/планшеты "--- массовые AR-устройства с камерами и датчиками движения. Их преимущество "--- доступность для обычных пользователей;
	\item Спецгарнитуры (HoloLens, Magic Leap) "--- профессиональные решения с расширенными возможностями трекинга и hands"=free управлением;
	\item Дополнительные сенсоры (LiDAR, проекторы) "--- повышают точность AR-систем и расширяют сферы применения.
\end{enumerate}

Развитие AR-оборудования направлено на улучшение точности, удобства и доступности технологии. Современные тенденции - миниатюризация устройств и расширение их функционала.



\conclusion

% Отобразить все источники. Даже те, на которые нет ссылок.
% \nocite{*}

\bibliographystyle{ugost2003}
\bibliography{thesis}

% Окончание основного документа и начало приложений Каждая последующая секция
% документа будет являться приложением
\appendix

\end{document}
